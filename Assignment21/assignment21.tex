\let\negmedspace\undefined
\let\negthickspace\undefined
\documentclass{article}
\usepackage{cite}
\usepackage{amsmath,amssymb,amsfonts,amsthm}
\usepackage{algorithmic}
\usepackage{graphicx}
\usepackage{textcomp}
\usepackage{xcolor}
\usepackage{txfonts}
\usepackage{listings}
\usepackage{enumitem}
\usepackage{mathtools}
\usepackage{gensymb}
\usepackage[breaklinks=true]{hyperref}
\usepackage{tkz-euclide} % loads  TikZ and tkz-base
\usepackage{listings}
\usepackage{gvv}
%
%\usepackage{setspace}
%\usepackage{gensymb}
%\doublespacing
%\singlespacing

%\usepackage{graphicx}
%\usepackage{amssymb}
%\usepackage{relsize}
%\usepackage[cmex10]{amsmath}
%\usepackage{amsthm}
%\interdisplaylinepenalty=2500
%\savesymbol{iint}
%\usepackage{txfonts}
%\restoresymbol{TXF}{iint}
%\usepackage{wasysym}
%\usepackage{amsthm}
%\usepackage{iithtlc}
%\usepackage{mathrsfs}
%\usepackage{txfonts}
%\usepackage{stfloats}
%\usepackage{bm}
%\usepackage{cite}
%\usepackage{cases}
%\usepackage{subfig}
%\usepackage{xtab}
%\usepackage{longtable}
%\usepackage{multirow}
%\usepackage{algorithm}
%\usepackage{algpseudocode}
%\usepackage{enumitem}
%\usepackage{mathtools}
%\usepackage{tikz}
%\usepackage{circuitikz}
%\usepackage{verbatim}
%\usepackage{tfrupee}
%\usepackage{stmaryrd}
%\usetkzobj{all}
%    \usepackage{color}                                            %%
%    \usepackage{array}                                            %%
%    \usepackage{longtable}                                        %%
%    \usepackage{calc}                                             %%
%    \usepackage{multirow}                                         %%
%    \usepackage{hhline}                                           %%
%    \usepackage{ifthen}                                           %%
  %optionally (for landscape tables embedded in another document): %%
%    \usepackage{lscape}     
%\usepackage{multicol}
%\usepackage{chngcntr}
%\usepackage{enumerate}

%\usepackage{wasysym}
%\documentclass[conference]{IEEEtran}
%\IEEEoverridecommandlockouts
% The preceding line is only needed to identify funding in the first footnote. If that is unneeded, please comment it out.

\newtheorem{theorem}{Theorem}[section]
\newtheorem{problem}{Problem}
\newtheorem{proposition}{Proposition}[section]
\newtheorem{lemma}{Lemma}[section]
\newtheorem{corollary}[theorem]{Corollary}
\newtheorem{example}{Example}[section]
%\newtheorem{thm}{Theorem}[section] 
%\newtheorem{defn}[thm]{Definition}
%\newtheorem{algorithm}{Algorithm}[section]
%\newtheorem{cor}{Corollary}
\newcommand{\BEQA}{\begin{eqnarray}}
\newcommand{\EEQA}{\end{eqnarray}}
\theoremstyle{remark}
\newtheorem{rem}{Remark}

%\bibliographystyle{ieeetr}
\begin{document}
\title{Latex Assignment21}
\author{D.V.S. NIKHIL}
\date{04 September, 2023}
\maketitle
\section*{Ex 12.4.5}
Find adjoint of each of the matrices in \ref{prob:1} to \ref{prob:2}
\begin{enumerate}
\item $\myvec
{1 & 2 \\ 3 & 4}$ \label{prob:1}
\item $\myvec
{1 & -1 & 2 \\ 2 & 3 & 5 \\ -2 & 0 & 1}$ \label{prob:2}
\end{enumerate}
Verify $A(adj A)=(adj A)A=\abs{A}I$  in \ref{prob:3} and \ref{prob:4}
\begin{enumerate}[resume]
\item $\myvec
{2 & 3 \\ -4 & -6}$ \label{prob:3}
\item $\myvec
{1 & -1 & 2 \\ 3 & 0 & -2 \\ 1 & 0 & 3}$ \label{prob:4}
\end{enumerate}
Find the inverse of each of the matrices (if it exists) given in \ref{prob:5} to \ref{prob:11}
\begin{enumerate}[resume]
\item $\myvec
{2 & 2 \\ 4 & 3}$ \label{prob:5}
\item $\myvec
{-1 & 5 \\ -3 & 2}$
\item $\myvec
{1 & 2 & 3 \\ 0 & 2 & 4 \\ 0 & 0 & 5}$
\item $\myvec
{1 & 0 & 0 \\ 3 & 3 & 0 \\ 5 & 2 & 1}$
\item $\myvec
{2 & 1 & 3 \\ 4 & -1 & 0 \\ -7 & 2 & 1}$
\item $\myvec
{1 & -1 & 2 \\ 0 & 2 & -3 \\ 3 & -2 & 4}$
\item $\myvec
{1 & 0 & 0 \\ 0 & \cos \alpha & \sin \alpha \\ 0 & \sin \alpha & -\cos \alpha}$ \label{prob:11}
\end{enumerate}
\begin{enumerate}[resume]
\item Let $A=\myvec
{3 & 7 \\ 2 & 5}$ and $B=\myvec
{6 & 8 \\ 7 & 9}$.Verify that $(AB)^{-1}=B^{-1} A^{-1}$.
\item Let $A=\myvec
 {3 & 1 \\ -1 & 2}$, show that $A^2-5A+7I=0$. Hence find $A^{-1}$.
\item For the matrix $A=\myvec
{3 & 2 \\ 1 & 1}$, find the numbers $a$ and $b$ such that $A^2+aA+bI=0$.
\item For the matrix $A=\myvec
{1 & 1 & 1 \\ 1 & 2 & 3 \\ 2 & -1 & 3}$. Show that $A^3-6A^2+5A+11I=0$. Hence, find $A^{-1}$.
\item If $A=\myvec
{2 & -1 & 1 \\ -1 & 2 & -1 \\ 1 & -1 & 2}$. Verify that $A^3-6A^2+9A-4I=0$ and hence find $A^{-1}$.
\item Let $A$ be a nonsingular square matrix of order $3\times 3$. Then $\abs{adj A}$ is equal to:
\begin{enumerate}
\item $\abs{A}$
\item $\abs{A^2}$
\item $\abs{A^3}$
\item $\abs{3A}$
\end{enumerate}
\item If $A$ is an invertible matrix of order $2$, then $\det(A^{-1})$ is equal to:
\begin{enumerate}
\item $\det(A)$
\item $\frac{1}{\det(A)}$
\item $1$
\item $0$
\end{enumerate}
\end{enumerate}
\end{document} 
