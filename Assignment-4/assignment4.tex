\let\negmedspace\undefined
\let\negthickspace\undefined
%\documentclass[journal,12pt,twocolumn]{IEEEtran}
\documentclass{article}
\usepackage{enumitem}
\usepackage{amsmath}
\usepackage{cite}
\usepackage{amsmath,amssymb,amsfonts,amsthm}
\usepackage{algorithmic}
\usepackage{graphicx}
\usepackage{textcomp}
\usepackage{xcolor}
\usepackage{txfonts}
\usepackage{listings}
\usepackage{enumitem}
\usepackage{mathtools}
\usepackage{gensymb}
\usepackage[breaklinks=true]{hyperref}
\usepackage{tkz-euclide} % loads  TikZ and tkz-base
\usepackage{listings}
%\usepackage{gvv}
%
%\usepackage{setspace}
%\usepackage{gensymb}
%\doublespacing
%\singlespacing

%\usepackage{graphicx}
%\usepackage{amssymb}
%\usepackage{relsize}
%\usepackage[cmex10]{amsmath}
%\usepackage{amsthm}
%\interdisplaylinepenalty=2500
%\savesymbol{iint}
%\usepackage{txfonts}
%\restoresymbol{TXF}{iint}
%\usepackage{wasysym}
%\usepackage{amsthm}
%\usepackage{iithtlc}
%\usepackage{mathrsfs}
%\usepackage{txfonts}
%\usepackage{stfloats}
%\usepackage{bm}
%\usepackage{cite}
%\usepackage{cases}
%\usepackage{subfig}
%\usepackage{xtab}
%\usepackage{longtable}
%\usepackage{multirow}
%\usepackage{algorithm}
%\usepackage{algpseudocode}
%\usepackage{enumitem}
%\usepackage{mathtools}
%\usepackage{tikz}
%\usepackage{circuitikz}
%\usepackage{verbatim}
%\usepackage{tfrupee}
%\usepackage{stmaryrd}
%\usetkzobj{all}
%    \usepackage{color}                                            %%
%    \usepackage{array}                                            %%
%    \usepackage{longtable}                                        %%
%    \usepackage{calc}                                             %%
%    \usepackage{multirow}                                         %%
%    \usepackage{hhline}                                           %%
%    \usepackage{ifthen}                                           %%
  %optionally (for landscape tables embedded in another document): %%
%    \usepackage{lscape}     
%\usepackage{multicol}
%\usepackage{chngcntr}
%\usepackage{enumerate}

%\usepackage{wasysym}
%\documentclass[conference]{IEEEtran}
%\IEEEoverridecommandlockouts
% The preceding line is only needed to identify funding in the first footnote. If that is unneeded, please comment it out.

\newtheorem{theorem}{Theorem}[section]
\newtheorem{problem}{Problem}
\newtheorem{proposition}{Proposition}[section]
\newtheorem{lemma}{Lemma}[section]
\newtheorem{corollary}[theorem]{Corollary}
\newtheorem{example}{Example}[section]
\newtheorem{definition}[problem]{Definition}
%\newtheorem{thm}{Theorem}[section] 
%\newtheorem{defn}[thm]{Definition}
%\newtheorem{algorithm}{Algorithm}[section]
%\newtheorem{cor}{Corollary}
\newcommand{\BEQA}{\begin{eqnarray}}
\newcommand{\EEQA}{\end{eqnarray}}
\newcommand{\define}{\stackrel{\triangle}{=}}
\theoremstyle{remark}
\newtheorem{rem}{Remark}

\begin{document}
\title{Latex Assignment4}
\author{D.V.S. NIKHIL}
\date{18 August,2023}
\maketitle
\section*{Exercise 10.3.5}
\begin{enumerate}
\item Which of the following pairs of linear equations has unique solution, no solution or infinitely many solutions.In case there is a unique solution,find it by using cross multiplication method:
  \begin{enumerate}[label=(\roman*)]
	\item \begin{align}
	x-3y-3=0\\
	3x-9y-2=0
        \end{align}
       \item \begin{align}
	2x+y=5\\
	3x+2y=8
	\end{align}
	\item \begin{align}
	3x-5y=20\\
	6x-10y=40
	\end{align}
	\item \begin{align}
	x-3y-7=0\\
	3x-3y-15=0
        \end{align}
 \end{enumerate}
\item 
  \begin{enumerate}[label=(\roman*)]
  \item For which values of $a$ and $b$ does the following pair of linear equations have an infinite number of solutions?
	\begin{align}
	2x+3y=7\\
	(a-b)x+(a-b)y=3a+b-2
	\end{align}
  \item For which value of $k$ will the following pair of linear equation have no solution?
	\begin{align}
	3x+y=1\\
	(2k-1)x+(k-1)y=2k+1
	\end{align}
   \end{enumerate}
\item Solve the following pair of linear equations by the substituions and cross multiplication method:
\begin{align}
8x+5y=9
\\ 3x+2y=4
\end{align}
\item Form the pair of linear equations in the following problems and find their solutions by any algebraic method:
\begin{enumerate}[label=(\roman*)]
\item A part of monthly hostel charges is fixed and the remaining depends on the number of days one has taken food in the mess. When a student $A$ takes food for $20$ days she has to pay Rs.$1000$ as hostel charges whereas a student $B$ who takes food for $26$ days, pays Rs.$1180$ as hostel charges. Find the fixed charges and the cost of food per day.
\item A fraction becomes $\frac{1}{3}$ when $1$ is subtracted from the numerator and it becomes $\frac{1}{4}$ when $8$ is added to the denominator. Find the fraction.
\item Yash scored $40$ marks in a test, getting $3$ marks for each right answer and losing $1$ mark for each wrong answer. Had $4$ marks been awarded for each correct answer and $2$ marks been deducted for each incorrect answer, then Yash would have scored $50$ marks. How many questions were there in the test?
\item Places $A$ and $B$ are $100km$ apart on a highway. One car starts from $A$ and another from $B$ at the same time. If the car travel in the same direction at different speeds, they meet in $5$hrs. If they travel towards each other, they meet in $1$hr. What are the speeds of the two cars?
\item The area of rectangle gets reduced by $9$ square units, if its length is reduced by $5$ units and breadth is increased by $3$ units. If we increase the length by $3$ units and the breadth by $2$ units, the area increases by $67$ square units. Find the dimensions of the rectangle.
\end{enumerate}
\end{enumerate}
\end{document} 
