\let\negmedspace\undefined
\let\negthickspace\undefined
%\documentclass[journal,12pt,twocolumn]{IEEEtran}
\documentclass{article}
\usepackage{cite}
\usepackage{amsmath,amssymb,amsfonts,amsthm}
\usepackage{algorithmic}
\usepackage{graphicx}
\usepackage{textcomp}
\usepackage{xcolor}
\usepackage{txfonts}
\usepackage{listings}
\usepackage{enumitem}
\usepackage{mathtools}
\usepackage{gensymb}
\usepackage[breaklinks=true]{hyperref}
\usepackage{tkz-euclide} % loads  TikZ and tkz-base
\usepackage{listings}
%\usepackage{gvv}
%
%\usepackage{setspace}
%\usepackage{gensymb}
%\doublespacing
%\singlespacing

%\usepackage{graphicx}
%\usepackage{amssymb}
%\usepackage{relsize}
%\usepackage[cmex10]{amsmath}
%\usepackage{amsthm}
%\interdisplaylinepenalty=2500
%\savesymbol{iint}
%\usepackage{txfonts}
%\restoresymbol{TXF}{iint}
%\usepackage{wasysym}
%\usepackage{amsthm}
%\usepackage{iithtlc}
%\usepackage{mathrsfs}
%\usepackage{txfonts}
%\usepackage{stfloats}
%\usepackage{bm}
%\usepackage{cite}
%\usepackage{cases}
%\usepackage{subfig}
%\usepackage{xtab}
%\usepackage{longtable}
%\usepackage{multirow}
%\usepackage{algorithm}
%\usepackage{algpseudocode}
%\usepackage{enumitem}
%\usepackage{mathtools}
%\usepackage{tikz}
%\usepackage{circuitikz}
%\usepackage{verbatim}
%\usepackage{tfrupee}
%\usepackage{stmaryrd}
%\usetkzobj{all}
%    \usepackage{color}                                            %%
%    \usepackage{array}                                            %%
%    \usepackage{longtable}                                        %%
%    \usepackage{calc}                                             %%
%    \usepackage{multirow}                                         %%
%    \usepackage{hhline}                                           %%
%    \usepackage{ifthen}                                           %%
  %optionally (for landscape tables embedded in another document): %%
%    \usepackage{lscape}     
%\usepackage{multicol}
%\usepackage{chngcntr}
%\usepackage{enumerate}

%\usepackage{wasysym}
%\documentclass[conference]{IEEEtran}
%\IEEEoverridecommandlockouts
% The preceding line is only needed to identify funding in the first footnote. If that is unneeded, please comment it out.

\newtheorem{theorem}{Theorem}[section]
\newtheorem{problem}{Problem}
\newtheorem{proposition}{Proposition}[section]
\newtheorem{lemma}{Lemma}[section]
\newtheorem{corollary}[theorem]{Corollary}
\newtheorem{example}{Example}[section]
\newtheorem{definition}[problem]{Definition}
%\newtheorem{thm}{Theorem}[section] 
%\newtheorem{defn}[thm]{Definition}
%\newtheorem{algorithm}{Algorithm}[section]
%\newtheorem{cor}{Corollary}
\newcommand{\BEQA}{\begin{eqnarray}}
\newcommand{\EEQA}{\end{eqnarray}}
\newcommand{\define}{\stackrel{\triangle}{=}}
\theoremstyle{remark}
\newtheorem{rem}{Remark}

%\bibliographystyle{ieeetr}
\begin{document}
\title{Latex Assignment3}
\author{D.V.S. NIKHIL}
\date{17 August,2023}
\maketitle
\section*{Exercise 10.3.2}
\begin{enumerate}
\item Form the pair of linear equations in the following problems and find their solutions graphically:
\begin{enumerate}[label=(\roman*)]
\item $10$ students of Class X took part in a Mathematics quiz.If the number of girls is $4$ more than the number of boys, find the number of boys and girls who took part in the quiz.
\item $5$ pencils and $7$ pens together cost $Rs.50$ whereas $7$ pencils and $5$ pens together cost $Rs.46$. Find the cost of one pencil and that of one pen.
\end{enumerate}
\item On comparing the ratios $\frac{a_{1}}{a_2},\frac{b_1}{b_2} and \frac{c_1}{c_2}$,find out whether the lines representing the following pairs of linear equations intersect at a point, are parallel or coincident:
\begin{enumerate}[label=(\roman*)]
\item \begin{align}
	5x-4y+8=0\\ 
      	7x+6y-9=0
	\end{align}
\item \begin{align}
	9x+3y+12=0\\
	18x+6y+24=0
	\end{align}
\item \begin{align}
        6x-3y+10=0\\
	2x-y+9=0
	\end{align}
\end{enumerate}
\item On comparing the ratios $\frac{a_1}{a_2},\frac{b_1}{b_2}$ and $\frac {c_1}{c_2}$,find out whether the following equations are consistent, or inconsistent:
\begin{enumerate}[label=(\roman*)]
	\item \begin{align}
       		3x+2y=5; \\
      		2x-3y=7
    	       \end{align} 
	\item \begin{align}
		2x-3y=8;\\
		4x-6y=9
		\end{align}
	\item \begin{align}
		\frac{3}{2}x+\frac{5}{3}y=7;\\
		9x-10y=14
		\end{align}
\item \begin{align}
		5x-3y=11;\\
		-10x+6y=-22
	\end{align}
\item \begin{align}
         \frac{4}{3}x+2y=8;\\
	  2x+3y=12
	\end{align}
\end{enumerate}
\item Which of the following pairs of linear equations are consistent/inconsistent? If consistent, obtain solution graphically:
\begin{enumerate}[label=(\roman*)]
\item \begin{align}
	x+y=5;\\
	2x+y=10
	\end{align}
\item \begin{align}
	x-y=8;\\
	3x-3y=16
	\end{align}
\item \begin{align}
 	2x+y-6=0;\\
 	4x-2y+4=0
	\end{align}
\item \begin{align}
	2x-2y-2=0;\\
	4x-4y-5=0
	\end{align}
\end{enumerate}
\item Half the perimeter of a rectangular garden,whose length is $4m$,more than its width,is $36m$. Find the dimensions of the garden.
\item Given the linear equation $2x+3y-8=0$,write another linear equation in two variables such that geometrical representation of the pair so formed is:
\begin{enumerate}[label=(\roman*)]
\item intersecting lines
\item parallel lines
\item coincident lines
\end{enumerate}
\item Draw the graphs of the equations $x-y+1=0$ and $3x+2y-12=0$.Determine the coordinates of the vertices of the triangle formed by these lines and the axis and shade the triangular region.
\end{enumerate}
\end{document}
