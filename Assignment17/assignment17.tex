\let\negmedspace\undefined
\let\negthickspace\undefined
\documentclass{article}
\usepackage{cite}
\usepackage{amsmath,amssymb,amsfonts,amsthm}
\usepackage{algorithmic}
\usepackage{graphicx}
\usepackage{textcomp}
\usepackage{xcolor}
\usepackage{txfonts}
\usepackage{listings}
\usepackage{enumitem}
\usepackage{tfrupee}
\usepackage{mathtools}
\usepackage{gensymb}
\usepackage[breaklinks=true]{hyperref}
\usepackage{tkz-euclide} % loads  TikZ and tkz-base
\usepackage{listings}
\usepackage{gvv}
%
%\usepackage{setspace}
%\usepackage{gensymb}
%\doublespacing
%\singlespacing

%\usepackage{graphicx}
%\usepackage{amssymb}
%\usepackage{relsize}
%\usepackage[cmex10]{amsmath}
%\usepackage{amsthm}
%\interdisplaylinepenalty=2500
%\savesymbol{iint}
%\usepackage{txfonts}
%\restoresymbol{TXF}{iint}
%\usepackage{wasysym}
%\usepackage{amsthm}
%\usepackage{iithtlc}
%\usepackage{mathrsfs}
%\usepackage{txfonts}
%\usepackage{stfloats}
%\usepackage{bm}
%\usepackage{cite}
%\usepackage{cases}
%\usepackage{subfig}
%\usepackage{xtab}
%\usepackage{longtable}
%\usepackage{multirow}
%\usepackage{algorithm}
%\usepackage{algpseudocode}
%\usepackage{enumitem}
%\usepackage{mathtools}
%\usepackage{tikz}
%\usepackage{circuitikz}
%\usepackage{verbatim}
%\usepackage{tfrupee}
%\usepackage{stmaryrd}
%\usetkzobj{all}
%    \usepackage{color}                                            %%
%    \usepackage{array}                                            %%
%    \usepackage{longtable}                                        %%
%    \usepackage{calc}                                             %%
%    \usepackage{multirow}                                         %%
%    \usepackage{hhline}                                           %%
%    \usepackage{ifthen}                                           %%
  %optionally (for landscape tables embedded in another document): %%
%    \usepackage{lscape}     
%\usepackage{multicol}
%\usepackage{chngcntr}
%\usepackage{enumerate}

%\usepackage{wasysym}
%\documentclass[conference]{IEEEtran}
%\IEEEoverridecommandlockouts
% The preceding line is only needed to identify funding in the first footnote. If that is unneeded, please comment it out.

\newtheorem{theorem}{Theorem}[section]
\newtheorem{problem}{Problem}
\newtheorem{proposition}{Proposition}[section]
\newtheorem{lemma}{Lemma}[section]
\newtheorem{corollary}[theorem]{Corollary}
\newtheorem{example}{Example}[section]
%\newtheorem{thm}{Theorem}[section] 
%\newtheorem{defn}[thm]{Definition}
%\newtheorem{algorithm}{Algorithm}[section]
%\newtheorem{cor}{Corollary}
\newcommand{\BEQA}{\begin{eqnarray}}
\newcommand{\EEQA}{\end{eqnarray}}
\theoremstyle{remark}
\newtheorem{rem}{Remark}

%\bibliographystyle{ieeetr}
\begin{document}
\title{Latex Assignment17}
\author{D.V.S. NIKHIL}
\date{31 August, 2023}
\maketitle
\section*{Ex 12.3.2}
\begin{enumerate}
\item Let $A= \myvec
 {2 & 4 \\ 3 & 2}, B=\myvec
 {1 & 3 \\ -2 & 5} , C=\myvec
{-2 & 5 \\ 3 & 4}$.
 Find each of the following:
\begin{enumerate}[label=(\roman*)]
\item $A+B$
\item $A-B$
\item $3A-C$
\item $AB$
\item $BA$
\end{enumerate}
\item Compute the following:
\begin{enumerate}[label=(\roman*)]
\item $\myvec  
{a & b \\ -b & a}+
\myvec{a & b \\ -b & a}$
\item $\myvec
{a^2+b^2 & b^2+c^2 \\ a^2+c^2 & a^2+b^2}+\myvec
{2ab & 2ac \\ -2ac & -2ab}$
\item $\myvec
{-1 & 4 & -6 \\ 8 & 5 & 16 \\ 2 & 8 & 5}+\myvec
{12 & 7 & 6 \\ 8 & 0 & 5 \\ 3 & 2 & 4}$
\item $\myvec
{\cos^2x & \sin^2x \\ \sin^2x & \cos^2x}+ \myvec{\sin^2x & \cos^2x \\ \cos^2x & \sin^2x}$
\end{enumerate}
\item Compute the following products:
\begin{enumerate}[label=(\roman*)]
\item $\myvec
{a & b  \\ b & -a} \myvec
{a & -b \\ b & a}$
\item $\myvec
{1 \\ 2 \\ 3 } \myvec
{2 & 3 & 4}$
\item $\myvec
{2 & 3 & 4 \\ 3 & 4 & 5 \\ 4 & 5 & 6 } \myvec
{1 & -3 & 5 \\ 0 & 2 & 4 \\ 5 & 0 & 5}$
\item  $\myvec
{2 & 1 \\ 3 & 2 \\ -1 & 1}\myvec
{1 & 0 & 1 \\ -1 & 2 & 1}$
\item $\myvec
{3 & -1 & 3 \\ 1 & 0 & 2 } \myvec
{2 & -3 \\ 1 & 0 \\ 3 & 1}$
\end{enumerate}
\item If $A=\myvec
{1 & 2 & -3 \\ 5 & 0 & 2 \\ 1 & -1 & 1}, B=\myvec
{3 & -1 & 2 \\ 4 & 2 & 5 \\ 2 & 0 & 3}$ and $C=\myvec
{4 & 1 & 2 \\ 0 & 3 & 2 \\ 1 & -2 & 3}$, then compute $(A+B)$ and $(B+C)$. Also, verify
that $A+(B-C)=(A+B)-C$.
\item If $A=\myvec
{\frac{2}{3} & 1 & \frac{5}{3} \\ \frac{1}{3} & \frac{2}{3} & \frac{4}{3} \\ \frac{7}{3}& 2 & \frac{2}{3}}$  and $B= \myvec
{\frac{2}{5} & \frac{3}{5} & 1 \\ \frac{1}{5} & \frac{2}{5} & \frac{4}{5} \\ \frac{7}{5} & \frac{6}{5} & \frac{2}{5}}$, then compute $3A-5B$.
\item Simplify $\cos \theta \myvec
{\cos \theta & \sin \theta \\ -\sin \theta & \cos \theta} + \myvec{\sin \theta & -\cos \theta \\ \cos \theta & \sin \theta}$.
\item Find $X$ and $Y$, if:
\begin{enumerate}[label=(\roman*)]
\item $X+Y= \myvec
{7 & 0 \\ 2 & 5}$  and $X-Y=\myvec
{3 & 0 \\ 0 & 4}$.
\item $2X+3Y=\myvec
{2 & 3 \\ 4 & 0}$ and $3X+2Y=\myvec
{2 & -2 \\ -1 & 5}$
\end{enumerate}
\item Find $X$, if $Y=\myvec
{3 & 2 \\ 1 & 4}$  and $2X+Y=\myvec
{1 & 0 \\ -3 & 2}$.
\item Find $x$ and $y$, if $2 \myvec
{1 & 3 \\ 0 & x}+ \myvec
{y & 0 \\ 1 & 2}= \myvec
{5 & 6 \\ 1 & 8}$.
\item Solve the equation for $x, y, z$ and $t$, if $2 \myvec
{x & y \\ z & t} + 3 \myvec
{1 & -1 \\ 0 & 2} = 3 \myvec
{3 & 5 \\ 4 & 6}$.
\item If $x \myvec
{2 \\ 3} + y \myvec
{-1 \\ 1}= \myvec
{10 \\ 5}$, find the values of $x$ and $y$.
\item Given $3 \myvec
{x & y \\ z & w}=\myvec
{x & 6 \\ -1 & 2w}+ \myvec
{4 & x+y \\ z+w & 3}$, find the values of $x, y, z$ and $w$.
\item If $F(x)=\myvec
{cosx & -sinx & 0 \\ sinx & cosx & 0 \\ 0 & 0 & 1}$ , show that $F(x)+F(y)=F(x+y)$.
\item Show that:
\begin{enumerate}[label=(\roman*)]
\item $\myvec 
{5 & -1 \\ 6 & 7} \myvec
{2 & 1 \\ 3 & 4} \neq \myvec
{2 & 1 \\ 3 & 4} \myvec
{5 & -1 \\ 6 & 7}$.
\item $\myvec
{1 & 2 & 3 \\ 0 & 1 & 0 \\ 1 & 1 & 0} \myvec
{-1 & 1 & 0 \\ 0 & -1 & 1 \\ 2 & 3 & 4} \neq \myvec
{-1 & 1 & 0 \\ 0 & -1 & 1 \\ 2 & 3 & 4} \myvec
{1 & 2 & 3 \\ 0 & 1 & 0 \\ 1 & 1 & 0}$
\end{enumerate}
\item Find $A^2-5A+6I$, if $A=\myvec{
2 & 0 & 1 \\ 2 & 1 & 3 \\ 1 & -1 & 0}$.
\item If $A=\myvec
{1 & 0 & 2 \\ 0 & 2 & 1 \\ 2 & 0 & 3}$, prove that $A^3-6A^2+7A+2I=0$.
\item If $A=\myvec
{3 & -2 \\ 4 & -2}$ and $I=\myvec
{1 & 0 \\ 0 & 1}$, find $k$ so that $A^2=kA-2I$.
\item If $A=\myvec
{0 & -\tan \frac{\alpha}{2} \\ \tan \frac{\alpha}{2} & 0}$  and $I$ is the identity matrix of order $2$, show that $I+A= (I-A) \myvec
{\cos \alpha & -\sin \alpha \\ \sin \alpha & \cos \alpha}$.
\item A trust fund has \rupee~30000 that must be invested in two different types of bonds. The first bomd pays $5\%$  interest per year, and the secon bond pays $7\%$ interest per year. Using matrix multiplication, detemine how to divide \rupee~30000 among the $2$ types of bonds. If the trust fund must obtain an annual total interest of:
\begin{enumerate}
\item \rupee~1800
\item \rupee~2000
\end{enumerate}
\item The bookshop of a particular school has $10$ dozen chemistry books, $8$ dozen phyaics books, $10$ dozen economics books. Their selling prices are \rupee~80, \rupee~60 and \rupee~40 each respectively. Find the total amount the bookshop will receive from selling all the books using matrix algebra.
\\ Assume $X, Y, Z,  W$ and $P$ are matrices of order $2\times n, 3\times k, 2\times p, n\times 3$  and $p\times k$, respectively. Choose the correct answer in  \ref{prob:21} and \ref{prob:22}.
\item The restriction on $n, k$ and $p$ so that $PY+WY$ will be defined are:\label{prob:21}
\begin{enumerate}
\item $k=3, p=n$
\item $k$ is arbitrary, $p=2$.
\item $p$ is arbitrary, $k=3$
\item $k=2, p=3$
\end{enumerate}
\item If $n=p$, then order of the matrix $7X-52$ is:\label{prob:22}
\begin{enumerate}
\item $p\times 2$
\item $2\times n$
\item $n\times 3$
\item $p\times n$
\end{enumerate}
\end{enumerate}
\end{document}  




  




