\let\negmedspace\undefined
\let\negthickspace\undefined
\documentclass{article}
\usepackage{cite}
\usepackage{amsmath,amssymb,amsfonts,amsthm}
\usepackage{algorithmic}
\usepackage{graphicx}
\usepackage{textcomp}
\usepackage{xcolor}
\usepackage{txfonts}
\usepackage{listings}
\usepackage{enumitem}
\usepackage{mathtools}
\usepackage{gensymb}
\usepackage[breaklinks=true]{hyperref}
\usepackage{tkz-euclide} % loads  TikZ and tkz-base
\usepackage{listings}
\usepackage{gvv}
%
%\usepackage{setspace}
%\usepackage{gensymb}
%\doublespacing
%\singlespacing

%\usepackage{graphicx}
%\usepackage{amssymb}
%\usepackage{relsize}
%\usepackage[cmex10]{amsmath}
%\usepackage{amsthm}
%\interdisplaylinepenalty=2500
%\savesymbol{iint}
%\usepackage{txfonts}
%\restoresymbol{TXF}{iint}
%\usepackage{wasysym}
%\usepackage{amsthm}
%\usepackage{iithtlc}
%\usepackage{mathrsfs}
%\usepackage{txfonts}
%\usepackage{stfloats}
%\usepackage{bm}
%\usepackage{cite}
%\usepackage{cases}
%\usepackage{subfig}
%\usepackage{xtab}
%\usepackage{longtable}
%\usepackage{multirow}
%\usepackage{algorithm}
%\usepackage{algpseudocode}
%\usepackage{enumitem}
%\usepackage{mathtools}
%\usepackage{tikz}
%\usepackage{circuitikz}
%\usepackage{verbatim}
%\usepackage{tfrupee}
%\usepackage{stmaryrd}
%\usetkzobj{all}
%    \usepackage{color}                                            %%
%    \usepackage{array}                                            %%
%    \usepackage{longtable}                                        %%
%    \usepackage{calc}                                             %%
%    \usepackage{multirow}                                         %%
%    \usepackage{hhline}                                           %%
%    \usepackage{ifthen}                                           %%
  %optionally (for landscape tables embedded in another document): %%
%    \usepackage{lscape}     
%\usepackage{multicol}
%\usepackage{chngcntr}
%\usepackage{enumerate}

%\usepackage{wasysym}
%\documentclass[conference]{IEEEtran}
%\IEEEoverridecommandlockouts
% The preceding line is only needed to identify funding in the first footnote. If that is unneeded, please comment it out.

\newtheorem{theorem}{Theorem}[section]
\newtheorem{problem}{Problem}
\newtheorem{proposition}{Proposition}[section]
\newtheorem{lemma}{Lemma}[section]
\newtheorem{corollary}[theorem]{Corollary}
\newtheorem{example}{Example}[section]
%\newtheorem{thm}{Theorem}[section] 
%\newtheorem{defn}[thm]{Definition}
%\newtheorem{algorithm}{Algorithm}[section]
%\newtheorem{cor}{Corollary}
\newcommand{\BEQA}{\begin{eqnarray}}
\newcommand{\EEQA}{\end{eqnarray}}
\theoremstyle{remark}
\newtheorem{rem}{Remark}

%\bibliographystyle{ieeetr}
\begin{document}
\title{Latex Assignment16}
\date{30 August, 2023}
\maketitle
\section*{Ex 12.3.1}
\begin{enumerate}
\item In the matrix $A=\myvec
{2 & 5 & 19 & -7\\
25 & -2 & \frac{5}{2} & 12\\
\sqrt 3 & 1 & -5 & 17}$, write:
\begin{enumerate}[label=(\roman*)]
\item The order of the matrix
\item The number of elements
\item Write the elements $a_{13}, a_{21}, a_{33}, a_{24}, a_{23}$
\end{enumerate}
\item If a matrix has $24$ elements, what are the possible order it can have? What if, it has $13$ elements?
\item If a matrix has $18$ elements, what are the possible orders it can have? What, if it has $5$ elements?
\item Construct a $2\times 2$ matrix, $A=[a_{ij}]$, whose elements are given by:
\begin{enumerate}[label=(\roman*)]
\item $a_ij=\frac{(i+j)^2}{2}$
\item $a_ij=\frac{i}{j}$
\item $a_ij=\frac{(i+2j)^2}{2}$
\end{enumerate}
\item Construct a $3\times 4$ matrix, whose elements are given by:
\begin{enumerate}[label=(\roman*)]
\item $a_ij=\frac{1}{2}\abs{-3i+j}$
\item $a_ij=2i-j$
\end{enumerate}
\item Find the values of $x, y$ and $z$ from the following equations:
\begin{enumerate}[label=(\roman*)]
\item $\myvec
{4 & 3 \\ x & 5} 
\myvec{y & z \\ 1 & 5}$
\item $\myvec
{x+y & 2 \\ 5+z & xy}
\myvec{ 6 &  2 \\  5  & 8}$
\item $\myvec
{x+y+z \\ x+z \\ y+z }
\myvec{9 \\ 5 \\  7}$
\end{enumerate}
\item Find the value of $a, b, c$ and $d$ from the equation:
\begin{align} \myvec 
{a-b & 2a-c \\ 2a-b &  3c+d } = 
\myvec{-1 & 5 \\ 0 & 13} \end{align}
\item $A=\myvec  [a_{ij}]_{m\times n\setminus}$ is a square matrix, if:
\begin{enumerate}
\item $m \le n$
\item $m \ge n$
\item $m=n$
\item None of these
\end{enumerate}
\item Which of the given values of $x$ and $y$ make the following pair of matrices equal:
\begin{align} \myvec
{3x+7 & 5 \\ y+1 & 2-3x}, 
\myvec{0 & y-2 \\ 8 & 4}\end{align}
\begin{enumerate}
\item $x=\frac{1}{3}, y=7$
\item Not possible to find 
\item $y=7, x=\frac{2}{3}$
\item $x=\frac{1}{3}, y=\frac{2}{3}$
\end{enumerate}
\item The number of all possible matrices of order $3\times 3$ with each entry $0$ or $1$ is:
\begin{enumerate}
\item $27$
\item $81$
\item $18$
\item $512$
\end{enumerate}
\end{enumerate}
\end{document}

 





