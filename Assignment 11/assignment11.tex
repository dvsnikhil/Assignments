
\let\negmedspace\undefined
\let\negthickspace\undefined
\documentclass{article}
\usepackage{cite}
\usepackage{amsmath,amssymb,amsfonts,amsthm}
\usepackage{algorithmic}
\usepackage{graphicx}
\usepackage{textcomp}
\usepackage{xcolor}
\usepackage{txfonts}
\usepackage{listings}
\usepackage{enumitem}
\usepackage{mathtools}
\usepackage{gensymb}
\usepackage[breaklinks=true]{hyperref}
\usepackage{tkz-euclide} % loads  TikZ and tkz-base
\usepackage{listings}
\usepackage{gvv}
%
%\usepackage{setspace}
%\usepackage{gensymb}
%\doublespacing
%\singlespacing

%\usepackage{graphicx}
%\usepackage{amssymb}
%\usepackage{relsize}
%\usepackage[cmex10]{amsmath}
%\usepackage{amsthm}
%\interdisplaylinepenalty=2500
%\savesymbol{iint}
%\usepackage{txfonts}
%\restoresymbol{TXF}{iint}
%\usepackage{wasysym}
%\usepackage{amsthm}
%\usepackage{iithtlc}
%\usepackage{mathrsfs}
%\usepackage{txfonts}
%\usepackage{stfloats}
%\usepackage{bm}
%\usepackage{cite}
%\usepackage{cases}
%\usepackage{subfig}
%\usepackage{xtab}
%\usepackage{longtable}
%\usepackage{multirow}
%\usepackage{algorithm}
%\usepackage{algpseudocode}
%\usepackage{enumitem}
%\usepackage{mathtools}
%\usepackage{tikz}
%\usepackage{circuitikz}
%\usepackage{verbatim}
%\usepackage{tfrupee}
%\usepackage{stmaryrd}
%\usetkzobj{all}
%    \usepackage{color}                                            %%
%    \usepackage{array}                                            %%
%    \usepackage{longtable}                                        %%
%    \usepackage{calc}                                             %%
%    \usepackage{multirow}                                         %%
%    \usepackage{hhline}                                           %%
%    \usepackage{ifthen}                                           %%
  %optionally (for landscape tables embedded in another document): %%
%    \usepackage{lscape}     
%\usepackage{multicol}
%\usepackage{chngcntr}
%\usepackage{enumerate}

%\usepackage{wasysym}
%\documentclass[conference]{IEEEtran}
%\IEEEoverridecommandlockouts
% The preceding line is only needed to identify funding in the first footnote. If that is unneeded, please comment it out.

\newtheorem{theorem}{Theorem}[section]
\newtheorem{problem}{Problem}
\newtheorem{proposition}{Proposition}[section]
\newtheorem{lemma}{Lemma}[section]
\newtheorem{corollary}[theorem]{Corollary}
\newtheorem{example}{Example}[section]
\newtheorem{definition}[problem]{Definition}
%\newtheorem{thm}{Theorem}[section] 
%\newtheorem{defn}[thm]{Definition}
%\newtheorem{algorithm}{Algorithm}[section]
%\newtheorem{cor}{Corollary}
\newcommand{\BEQA}{\begin{eqnarray}}
\newcommand{\EEQA}{\end{eqnarray}}
\theoremstyle{remark}
\newtheorem{rem}{Remark}

%\bibliographystyle{ieeetr}
\begin{document}
\title{Latex Assignment11}
\author{D.V.S. NIKHIL}
\date{27 August,2023}
\maketitle
\section*{Ex 11.10.1}
\begin{enumerate}
\item Draw a quadrilateral in the Cartesian plane, whose vertices are $(-4,5), (0,7), (5,-5)$ and $(-4,-2)$. Also, find its area.
\item The base of an equilateral triangle with side $2a$ lies along the y-axis such that the mid-point of the base is at the origin. Find vertices of the triangle.
\item Find the distance between $P(x_1,y_1),Q(x_2,y_2)$ when:
\begin{enumerate}[label=(\roman*)]
\item $PQ$ is parallel to the y-axis.
\item $PQ$ is parellel to the x-axis.
\end{enumerate}
\item Find the point x-axis, which is equidistant from the points $(7,6)$ and $(3,4)$.
\item Find the slope of a line, which passes through the origin, and the mid-point of the line segment joining the points $P(0,-4)$ and $B(8,0)$.
\item Without using the Pythagoras thorem, show that the points $(4,4), (3,5)$ and $(-1,-1)$ are the vertices of a right angled triangle.
\item Find the slope of the line, which makes an angle of 30° with the positive  direction of y-axis measured anticlockwise.
\item Find the value of $x$ for which the points $(x,-1), (2,1)$ and $(4,5)$ are collinear.
\item Without using distance formula, show that points $(-2,-1), (4,0), (3,3)$ and $(-3,2)$ are the vertices of the parallelogram.
\item Find the angle between the x-axis and the line joining the points $(3,-1)$ and $(4,-2)$.
\item The slope of a line is double of the slope of another line. If tangent of the angle between them is $\frac{!}{3}$, find the slopes of the lines.
\item A line passes through $(x_1,y_1)$ and $(h,k)$. If slope of the line is m, show that:
\item $k-y_1= m(h-x_1)$
\item If three points $(h,0), (a,b)$ and $(0,k)$ lie on a line, show that $\frac{a}{h}+\frac{b}{k}=1$.
\item Consider the following population and year graph  \figref{fig:10.10}, find the slope of the line $AB$ and using it, find what will be the population in the year $2010$?
\begin{figure}[ht]
\centering
\includegraphics[width=\columnwidth]{figs/10.10.png}
\caption{10.10}
\label{fig:10.10}
\end{figure}
\end{enumerate}
\end{document}
