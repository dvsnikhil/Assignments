\let\negmedspace\undefined
\let\negthickspace\undefined
\documentclass{article}
\usepackage{cite}
\usepackage{amsmath,amssymb,amsfonts,amsthm}
\usepackage{algorithmic}
\usepackage{graphicx}
\usepackage{textcomp}
\usepackage{xcolor}
\usepackage{txfonts}
\usepackage{listings}
\usepackage{enumitem}
\usepackage{mathtools}
\usepackage{gensymb}
\usepackage[breaklinks=true]{hyperref}
\usepackage{tkz-euclide} % loads  TikZ and tkz-base
\usepackage{listings}
\usepackage{gvv}
%
%\usepackage{setspace}
%\usepackage{gensymb}
%\doublespacing
%\singlespacing

%\usepackage{graphicx}
%\usepackage{amssymb}
%\usepackage{relsize}
%\usepackage[cmex10]{amsmath}
%\usepackage{amsthm}
%\interdisplaylinepenalty=2500
%\savesymbol{iint}
%\usepackage{txfonts}
%\restoresymbol{TXF}{iint}
%\usepackage{wasysym}
%\usepackage{amsthm}
%\usepackage{iithtlc}
%\usepackage{mathrsfs}
%\usepackage{txfonts}
%\usepackage{stfloats}
%\usepackage{bm}
%\usepackage{cite}
%\usepackage{cases}
%\usepackage{subfig}
%\usepackage{xtab}
%\usepackage{longtable}
%\usepackage{multirow}
%\usepackage{algorithm}
%\usepackage{algpseudocode}
%\usepackage{enumitem}
%\usepackage{mathtools}
%\usepackage{tikz}
%\usepackage{circuitikz}
%\usepackage{verbatim}
%\usepackage{tfrupee}
%\usepackage{stmaryrd}
%\usetkzobj{all}
%    \usepackage{color}                                            %%
%    \usepackage{array}                                            %%
%    \usepackage{longtable}                                        %%
%    \usepackage{calc}                                             %%
%    \usepackage{multirow}                                         %%
%    \usepackage{hhline}                                           %%
%    \usepackage{ifthen}                                           %%
  %optionally (for landscape tables embedded in another document): %%
%    \usepackage{lscape}     
%\usepackage{multicol}
%\usepackage{chngcntr}
%\usepackage{enumerate}

%\usepackage{wasysym}
%\documentclass[conference]{IEEEtran}
%\IEEEoverridecommandlockouts
% The preceding line is only needed to identify funding in the first footnote. If that is unneeded, please comment it out.

\newtheorem{theorem}{Theorem}[section]
\newtheorem{problem}{Problem}
\newtheorem{proposition}{Proposition}[section]
\newtheorem{lemma}{Lemma}[section]
\newtheorem{corollary}[theorem]{Corollary}
\newtheorem{example}{Example}[section]
\newtheorem{definition}[problem]{Definition}
%\newtheorem{thm}{Theorem}[section] 
%\newtheorem{defn}[thm]{Definition}
%\newtheorem{algorithm}{Algorithm}[section]
%\newtheorem{cor}{Corollary}
\newcommand{\BEQA}{\begin{eqnarray}}
\newcommand{\EEQA}{\end{eqnarray}}
%\newcommand{\define}{\stackrel{\triangle}{=}}
\theoremstyle{remark}
\newtheorem{rem}{Remark}

%\bibliographystyle{ieeetr}
\begin{document}
\title{Latex Assgnment10}
\author{D.V.S. NiKHIL}
\date{26 August, 2023}
\maketitle
\section*{Ex 10.7.4}
\begin{enumerate}
\item Determine the ratio in which the line $2x+y-4=0$ divides the line segment joining the points $A(2,-2)$  and $B(3,7)$.
\item Find a relation between $x$ and $y$ if the points $(x,y), (1,2)$ and $(7,0)$ are collinear.
\item Find the centre of a circle passing through the points $(6,-6), (3,-7)$ and $(3,3)$.
\item The two opposite vertices of a square are $(-1,2)$ and $(3,2)$. Find the coordinates of the two other vertices.
\item The Class X students of a secondary school in Krishinagar have been allotted a rectangular plot of land for their gardening activity. Sapling of Gulmohar are planted on the boundary at a distance of 1m from each other. there is a triangular grassy lawn in the plot as shown in \figref{fig:7.14}. The students are to sow seeds of flowering plants on the remaining area of the plot.
\begin{enumerate}[label=(\roman*)]
\item Taking $A$ as origin, find the coordinates of the vertices of the triangle.
\item What will be tthe coordinates of the vertices of $\triangle PQR$ if $C$ is the origin Also calculate the areas of the triangles in these cases. What do you observe?
\end{enumerate}
\begin{figure}[ht]
\centering
\includegraphics[width=\columnwidth]{figs/7.14.png}
\caption{7.14}
  \label{fig:7.14}
\end{figure}
\item The vertices of a $\triangle ABC$ are $A(4,6), B(1,5)$ and $C(7,2)$. A line is drawn to intersect sides $AB$ and $AC$ at $D$ and $E$ respectively, such that $\frac {AD}{AB}=\frac{AE}{AC}=\frac{1}{4}$. Calculate the area of the $\triangle AD$ and compare it with he area of $\triangle ABC$.
\item Let $A(4,2), B(6,5)$ and $C(1,4)$ be the vertices of $\triangle ABC$.
\begin{enumerate}[label=(\roman*)]
\item The median from $A$ meets $BC$ at $D$. Find the coordinates of the points $D$.
\item Find the coordinates of the point $P$ on $AD$ such that $AP:PD=2:1$.
\item Find the coordinates of points Q and R on medians $BE$ and $CF$ respectively such that $BQ:QE=2:1$ and $CR:RF=2:1$.
\item What do you observe?
\item If $A(x_1,y_1), B(x_2,y_2)$ and $C(x_3,y_3)$ are the vertices of $\triangle ABC$, find the coordinates of the triangle.
\end{enumerate}
\item  $ABCD$ is a rectangle formed by the points  $A(-1,-1), B(-1,4), C(5,4)$ and $D(5,-1)$. $P, Q, R$ and $S$ are the mid points of $AB, BC, CD$ and $DA$ respectively. Is the quadrilateral $PQRS$ a square? a rectangle? or a rhombus? Justify your answer.
\end{enumerate}
\end{document} 
