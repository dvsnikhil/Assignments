\let\negmedspace\undefined
\let\negthickspace\undefined
\documentclass{article}
\usepackage{cite}
\usepackage{amsmath,amssymb,amsfonts,amsthm}
\usepackage{algorithmic}
\usepackage{graphicx}
\usepackage{textcomp}
\usepackage{xcolor}
\usepackage{txfonts}
\usepackage{listings}
\usepackage{enumitem}
\usepackage{mathtools}
\usepackage{gensymb}
\usepackage[breaklinks=true]{hyperref}
\usepackage{tkz-euclide} % loads  TikZ and tkz-base
\usepackage{listings}
\usepackage{gvv}
%
%\usepackage{setspace}
%\usepackage{gensymb}
%\doublespacing
%\singlespacing

%\usepackage{graphicx}
%\usepackage{amssymb}
%\usepackage{relsize}
%\usepackage[cmex10]{amsmath}
%\usepackage{amsthm}
%\interdisplaylinepenalty=2500
%\savesymbol{iint}
%\usepackage{txfonts}
%\restoresymbol{TXF}{iint}
%\usepackage{wasysym}
%\usepackage{amsthm}
%\usepackage{iithtlc}
%\usepackage{mathrsfs}
%\usepackage{txfonts}
%\usepackage{stfloats}
%\usepackage{bm}
%\usepackage{cite}
%\usepackage{cases}
%\usepackage{subfig}
%\usepackage{xtab}
%\usepackage{longtable}
%\usepackage{multirow}
%\usepackage{algorithm}
%\usepackage{algpseudocode}
%\usepackage{enumitem}
%\usepackage{mathtools}
%\usepackage{tikz}
%\usepackage{circuitikz}
%\usepackage{verbatim}
%\usepackage{tfrupee}
%\usepackage{stmaryrd}
%\usetkzobj{all}
%    \usepackage{color}                                            %%
%    \usepackage{array}                                            %%
%    \usepackage{longtable}                                        %%
%    \usepackage{calc}                                             %%
%    \usepackage{multirow}                                         %%
%    \usepackage{hhline}                                           %%
%    \usepackage{ifthen}                                           %%
  %optionally (for landscape tables embedded in another document): %%
%    \usepackage{lscape}     
%\usepackage{multicol}
%\usepackage{chngcntr}
%\usepackage{enumerate}

%\usepackage{wasysym}
%\documentclass[conference]{IEEEtran}
%\IEEEoverridecommandlockouts
% The preceding line is only needed to identify funding in the first footnote. If that is unneeded, please comment it out.

\newtheorem{theorem}{Theorem}[section]
\newtheorem{problem}{Problem}
\newtheorem{proposition}{Proposition}[section]
\newtheorem{lemma}{Lemma}[section]
\newtheorem{corollary}[theorem]{Corollary}
\newtheorem{example}{Example}[section]
%\newtheorem{thm}{Theorem}[section] 
%\newtheorem{defn}[thm]{Definition}
%\newtheorem{algorithm}{Algorithm}[section]
%\newtheorem{cor}{Corollary}
\newcommand{\BEQA}{\begin{eqnarray}}
\newcommand{\EEQA}{\end{eqnarray}}
\theoremstyle{remark}
\newtheorem{rem}{Remark}

%\bibliographystyle{ieeetr}
\begin{document}
\title{Latex Assignment19}
\author{D.V.S. NIKHIL}
\date{02 Sep, 2023}
\maketitle
\section*{Ex 12.4.1}
Evaluate the determinants in \ref{prob:1} to \ref{prob:2}:
\begin{enumerate}
\item $\myvec
{2 & 4 \\ -5 & -1}$ \label{prob:1}
\item  \label{prob:2}
\begin{enumerate}[label=(\roman*)]
\item $\myvec
{\cos \theta  &  -\sin \theta \\ \sin \theta & \cos \theta}$
\item $\myvec
{x^2-x+1 & x-1 \\ x+1 & x+1}$ 
\end{enumerate}
\item If $A=\myvec
{1 & 2 \\ 4 & 2}$, then show that $\abs{2A}=\abs{4A}$.
\item If $A=\myvec
{1 & 0 & 1 \\ 0 & 1 & 2 \\ 0 & 0 & 4}$, then show that $\abs{3A}=\abs{27A}$.
\item Evaluate the determinants:
\begin{enumerate}[label=(\roman*)]
\item $\myvec
{3 & -1 & -2 \\ 0 & 0 & 1 \\ 3 & -5 & 0}$
\item $\myvec
{3 & -4 & 5 \\ 1 & 1 & -2 \\ 2 & 3 & 1}$
\item $\myvec
{0 & 1 & 2 \\ -1 & 0 & -3 \\ -2 & 3 & 0}$
\item $\myvec
{2 & -1 & -2 \\ 0 & 2 & -1 \\ 3 & -5 & 0}$
\end{enumerate}
\item If $A=\myvec
{1 & 1 & -2 \\ 2 & 1 & -3 \\ 5 & 4 & -9}$, find $A$.
\item Find values of $x$, if:
\begin{enumerate}[label=(\roman*)]
\item $\myvec
{2 & 4 \\ 5 & 1}=\myvec {2x & 4 \\ 6 & x}$
\item $\myvec
{2 & 3 \\ 4 & 5}=\myvec  {x & 3 \\ 2x & 5}$
\end{enumerate}
\item If $\myvec
{x & 2 \\ 18 & x}=\myvec {6 & 2 \\ 18 & 6}$, then $x$ is equal to:
\begin{enumerate}
\item $6$
\item $\pm 6$
\item $-6$
\item $0$
\end{enumerate}
\end{enumerate}
\end{document}


  




