\let\negmedspace\undefined
\let\negthickspace\undefined
\documentclass{article}
\usepackage{cite}
\usepackage{amsmath,amssymb,amsfonts,amsthm}
\usepackage{algorithmic}
\usepackage{graphicx}
\usepackage{textcomp}
\usepackage{xcolor}
\usepackage{txfonts}
\usepackage{listings}
\usepackage{enumitem}
\usepackage{mathtools}
\usepackage{gensymb}
\usepackage[breaklinks=true]{hyperref}
\usepackage{tkz-euclide} % loads  TikZ and tkz-base
\usepackage{listings}
\usepackage{gvv}
%
%\usepackage{setspace}
%\usepackage{gensymb}
%\doublespacing
%\singlespacing

%\usepackage{graphicx}
%\usepackage{amssymb}
%\usepackage{relsize}
%\usepackage[cmex10]{amsmath}
%\usepackage{amsthm}
%\interdisplaylinepenalty=2500
%\savesymbol{iint}
%\usepackage{txfonts}
%\restoresymbol{TXF}{iint}
%\usepackage{wasysym}
%\usepackage{amsthm}
%\usepackage{iithtlc}
%\usepackage{mathrsfs}
%\usepackage{txfonts}
%\usepackage{stfloats}
%\usepackage{bm}
%\usepackage{cite}
%\usepackage{cases}
%\usepackage{subfig}
%\usepackage{xtab}
%\usepackage{longtable}
%\usepackage{multirow}
%\usepackage{algorithm}
%\usepackage{algpseudocode}
%\usepackage{enumitem}
%\usepackage{mathtools}
%\usepackage{tikz}
%\usepackage{circuitikz}
%\usepackage{verbatim}
%\usepackage{tfrupee}
%\usepackage{stmaryrd}
%\usetkzobj{all}
%    \usepackage{color}                                            %%
%    \usepackage{array}                                            %%
%    \usepackage{longtable}                                        %%
%    \usepackage{calc}                                             %%
%    \usepackage{multirow}                                         %%
%    \usepackage{hhline}                                           %%
%    \usepackage{ifthen}                                           %%
  %optionally (for landscape tables embedded in another document): %%
%    \usepackage{lscape}     
%\usepackage{multicol}
%\usepackage{chngcntr}
%\usepackage{enumerate}

%\usepackage{wasysym}
%\documentclass[conference]{IEEEtran}
%\IEEEoverridecommandlockouts
% The preceding line is only needed to identify funding in the first footnote. If that is unneeded, please comment it out.

\newtheorem{theorem}{Theorem}[section]
\newtheorem{problem}{Problem}
\newtheorem{proposition}{Proposition}[section]
\newtheorem{lemma}{Lemma}[section]
\newtheorem{corollary}[theorem]{Corollary}
\newtheorem{example}{Example}[section]
\newtheorem{definition}[problem]{Definition}
%\newtheorem{thm}{Theorem}[section] 
%\newtheorem{defn}[thm]{Definition}
%\newtheorem{algorithm}{Algorithm}[section]
%\newtheorem{cor}{Corollary}
\newcommand{\BEQA}{\begin{eqnarray}}
\newcommand{\EEQA}{\end{eqnarray}}
\theoremstyle{remark}
\newtheorem{rem}{Remark}

%\bibliographystyle{ieeetr}
\begin{document}
\title{Latex Assignment12}
\author{D.V.S. NIKHIL}
\date{28 August, 2023}
\maketitle
\section*{Ex 11.10.3}
\begin{enumerate}
\item Reduce the following equations into slope-intercept form and find their slopes and the y-intercepts:
\begin{enumerate}[label=(\roman*)]
\item $x+7y=0$
\item $6x+3y-5=0$
\item $y=0$
\end{enumerate}
\item Reduce the following equations into intercept form and find their intercepts on the axes:
\begin{enumerate}[label=(\roman*)]
\item $3x+2y-12=0$
\item $4x-3y=6$
\item $3y+2=0$
\end{enumerate}
\item Reduce the folloeing equations into normal form. Find their perpendicular distances from the origin and angle between perpendicular and the positive x-axis:
\begin{enumerate}[label=(\roman*)]
\item $x-\sqrt3y+8=0$
\item $y-2=0$
\item $x-y=4$
\end{enumerate}
\item Find the distance of the point $(-1,1)$ from the line $12(x+6)=5(y-2)$.
\item Find the points on the x-axis, whose distances from the line $\frac{x}{3}+\frac{y}{4}=1$ are 4 units.
\item Find the distance between parallel lines:
\begin{enumerate}[label=(\roman*)]
           \item
 \begin{align}
        15x+8y-34=0 \text{ and } 15x+8y+31=0
              \end{align}
            \item
\begin{align}
        l(x+y)+p=0 \text{ and } l(x+y)-r=0
              \end{align}
\end{enumerate}
\item Find equation of the line parallel to the line $3x-4y+2=0$ and passing through the point $(-2,3)$.
\item Find equation of the line perpendicular to the line $x-7y+5=0$ and having $x$ intercept $3$.
\item Find angles between the lines $\sqrt3x+y=1$ and $x+\sqrt3y=1$.
\item The line through the points $(h,3)$ and $(4,1)$ intersects the line $7x-9y-19=0$ at right angle. Find the value of $h$.
\item Prove that the line through the point $(x,y)$ and parallel to the line $Ax+By+C=0$ is
\\ $A(x-x_1)+B(y-y_1)=0$.
\item Two lines passing through the point $(2,3)$ intersects each other at an angle at 60\degree. If the shape of one line is $2$, find equation of the other line.
\item Find the equation of the right bisector of the line segment joining the point $(3,4)$ and $(-1,2)$.
\item Find the coordinates of the foot of perpendicular4 from the point $(-1,3)$ to the line $3x+4y-16=0$
\item The perpendicular from the origin to the line $y=mx+c$ meets it at the point $(-1,2)$. Find the values of $m$ and $c$.
\item If $p$ and $q$ are the lengths of perpendiculars from the origin to the lines $x cos\theta-y sin\theta=k cos 2\theta$ and $x sec\theta+y cosec\theta=k$ respectively, prove that $p^2+4q^2=k^2$.
\item In the triangle $ABC$ ith vertices $A(2,3), B(4,-1)$ and $C(1,2)$, find the equation and length of altitude from the vertex $A$.
\item If $p$ is the length of perpendicular from the origin to the line whose intercepts on the axes are $a$  and $b$, then show that $\frac{1}{p^2}=\frac{1}{a^2}+\frac{1}{b^2}$.
\end{enumerate}
\end{document}  
