\let\negmedspace\undefined
\let\negthickspace\undefined
\documentclass{article}
\usepackage{cite}
\usepackage{amsmath,amssymb,amsfonts,amsthm}
\usepackage{algorithmic}
\usepackage{graphicx}
\usepackage{textcomp}
\usepackage{xcolor}
\usepackage{txfonts}
\usepackage{listings}
\usepackage{enumitem}
\usepackage{mathtools}
\usepackage{gensymb}
\usepackage[breaklinks=true]{hyperref}
\usepackage{tkz-euclide} % loads  TikZ and tkz-base
\usepackage{listings}
\usepackage{gvv}
%
%\usepackage{setspace}
%\usepackage{gensymb}
%\doublespacing
%\singlespacing

%\usepackage{graphicx}
%\usepackage{amssymb}
%\usepackage{relsize}
%\usepackage[cmex10]{amsmath}
%\usepackage{amsthm}
%\interdisplaylinepenalty=2500
%\savesymbol{iint}
%\usepackage{txfonts}
%\restoresymbol{TXF}{iint}
%\usepackage{wasysym}
%\usepackage{amsthm}
%\usepackage{iithtlc}
%\usepackage{mathrsfs}
%\usepackage{txfonts}
%\usepackage{stfloats}
%\usepackage{bm}
%\usepackage{cite}
%\usepackage{cases}
%\usepackage{subfig}
%\usepackage{xtab}
%\usepackage{longtable}
%\usepackage{multirow}
%\usepackage{algorithm}
%\usepackage{algpseudocode}
%\usepackage{enumitem}
%\usepackage{mathtools}
%\usepackage{tikz}
%\usepackage{circuitikz}
%\usepackage{verbatim}
%\usepackage{tfrupee}
%\usepackage{stmaryrd}
%\usetkzobj{all}
%    \usepackage{color}                                            %%
%    \usepackage{array}                                            %%
%    \usepackage{longtable}                                        %%
%    \usepackage{calc}                                             %%
%    \usepackage{multirow}                                         %%
%    \usepackage{hhline}                                           %%
%    \usepackage{ifthen}                                           %%
  %optionally (for landscape tables embedded in another document): %%
%    \usepackage{lscape}     
%\usepackage{multicol}
%\usepackage{chngcntr}
%\usepackage{enumerate}

%\usepackage{wasysym}
%\documentclass[conference]{IEEEtran}
%\IEEEoverridecommandlockouts
% The preceding line is only needed to identify funding in the first footnote. If that is unneeded, please comment it out.

\newtheorem{theorem}{Theorem}[section]
\newtheorem{problem}{Problem}
\newtheorem{proposition}{Proposition}[section]
\newtheorem{lemma}{Lemma}[section]
\newtheorem{corollary}[theorem]{Corollary}
\newtheorem{example}{Example}[section]
\newtheorem{definition}[problem]{Definition}
%\newtheorem{thm}{Theorem}[section] 
%\newtheorem{defn}[thm]{Definition}
%\newtheorem{algorithm}{Algorithm}[section]
%\newtheorem{cor}{Corollary}
\newcommand{\BEQA}{\begin{eqnarray}}
\newcommand{\EEQA}{\end{eqnarray}}
%\newcommand{\define}{\stackrel{\triangle}{=}}
\theoremstyle{remark}
\newtheorem{rem}{Remark}

%\bibliographystyle{ieeetr}
\begin{document}
\title{Latex Assignment9}
\author{D.V.S. NIKHIL}
\date{25 August,2023}
\maketitle
\section*{Exercise 10.7.2}
\begin{enumerate}
\item Find the coordinates of the point which divides the join of $(-1,7)$ and $(4,-3)$ in the ratio 2:3.
\item Find the coordinates of the points of trisection of the line segment joining $(4,-1)$ and $(-2,3)$.
\item To conduct Sports Day activities, in your rectangular shaped school ground ABCD, lines have been drawn with chalk powder at a distance of 1m each. $100$ flower pots have been placed at a distance of 1m from each other along AD, as shown in \figref{fig:7.12}. Niharika runs $\frac {1}{4}th$ distance AD on the 2nd line and posts a green flag. Preet runs $\frac {1}{5}th$ the distance AD on the eighth line and posts a red flag. What is the distance between both the flags? If Rashmi has to post a blue flag exactly halfway between the line segment joining the two flags, where should she post her flag?
\begin{figure}[ht]
\centering
\includegraphics[width=\columnwidth]{figs/7.12.png}
\caption{7.12}
  \label{fig:7.12}
\end{figure}
\item Find the ratio in which the line segment joining the points $(-3,10)$ and $(6,-8)$ is divided by $(1,-6)$.
\item Find the ratio in which the line segment joining $A(1,-5)$ and $B(-4,5)$ is divided by the x-axis. Also find the coordinates of the point of division.
\item If $(1,2), (4,y), (x,6)$ and $(3,5)$ are the vertices of parallelogram taken in order, find x and y.
\item Find the coordinates of a point A, where AB is the diameter of a circle whose centre is $(2,-3)$ and B is $(1,4)$.
\item If A and B are $(-2,-2)$ and $(2,-4)$ respectively, find the coordinates of P such that AP= $\frac {3}{7}$ AB  and P lies on the line segment AB.
\item Find the coordinates of the points which divide the line segment joining A $(2,-2)$ and B $(2,8)$ into four equal parts.
\item Find the area of a rhombus if its vertices are $(3,0), (4,5), (1,-4)$ and $(-2,-1)$ taken in order.
\end{enumerate}
\end{document} 
